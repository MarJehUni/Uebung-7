\documentclass[10pt]{article}
\usepackage{ulem}
\usepackage[utf8]{inputenc}
\usepackage[T1]{fontenc}
\usepackage{lmodern}
\usepackage{graphicx}
\usepackage{amsmath} 
\usepackage{amssymb}
\usepackage[ngerman]{babel}
\usepackage{blindtext}
\usepackage[left=2.5cm,right=2.5cm,top=2.5cm,bottom=2.5cm]{geometry}


\begin{document}
\begin{titlepage}
\title{Mein Latex Dokument}
\author{Martin Jehle}
\date{\today}
\maketitle

\end{titlepage}
\bigskip

\noindent \Huge 1. Das ist der erste Abschnitt\\
\normalsize \medskip

\noindent Hier kann auch blablabla stehen.
\bigskip

\noindent \Huge 2. Tabelle
\normalsize \medskip
\begin{center}
\begin{tabular}{c|c|c|c}
 & Punkte erhalten & Punkte möglich & \% \\ 
\hline 
Aufgabe 1 & 2 & 1 & 50 \\ 
Aufgabe 2 & 3 & 3 & 100 \\ 
Aufgabe 3 & 3 & 3 & 100 \\ 
\end{tabular}\medskip

Tabelle 1: Diese Tabelle kann auch Werte enthalten 
\end{center}
\bigskip

\noindent \Huge 3. Formeln \normalsize \medskip

\noindent \Large a) Pythagoras\\
\medskip \normalsize

\noindent Der Satz des Pythagoras errechnet sich wie folgt: $a^2+b2=c^2$. Daraus können wir die Länge der Hypothenuse wie folgt berechnen
$c=\sqrt{a^2+b^2}$.
\medskip

\noindent \Large b) Summen\\
\medskip \normalsize

\noindent Wir können auch die Formel für eine Summe angeben:
\begin{equation}
 s=\sum_{i=1}^ni = \frac{n \cdot (n+1)}{2}
\end{equation}
\end{document}